\documentclass[12pt]{article}
\usepackage[margin=1in]{geometry} 
\usepackage{amsmath,amsthm,amssymb,amsfonts}
\usepackage{listings}
 
\newcommand{\N}{\mathbb{N}}
\newcommand{\Z}{\mathbb{Z}}
 
\newenvironment{problem}[2][Problem]{\begin{trivlist}
\item[\hskip \labelsep {\bfseries #1}\hskip \labelsep {\bfseries #2.}]}{\end{trivlist}}
%If you want to title your bold things something different just make another thing exactly like this but replace "problem" with the name of the thing you want, like theorem or lemma or whatever
 
\begin{document}
 
%\renewcommand{\qedsymbol}{\filledbox}
%Good resources for looking up how to do stuff:
%Binary operators: http://www.access2science.com/latex/Binary.html
%General help: http://en.wikibooks.org/wiki/LaTeX/Mathematics
%Or just google stuff
 
\title{Assignment 2\\ MEEN 357}
\author{Jacob Hartzer}
\maketitle
 
 \section*{Task 6}
\begin{problem} {1} Machine Epsilon

Machine epsilon for this 9-bit floating-point number would be half the difference between the following floats: 000000000 and 000000001. This can be found as:
\begin{align*}
\epsilon &= \frac{-1^0 * 2^{1-3} * 0.00001 - -1^0 * 2^{1-3} * 0.00000}{2} \\
& = \frac{(0.0000001 – 0.0000000)}{2}\\
& = 0.00000001_2\\
&= 2^{-8}\\
&=0.00390625
\end{align*}

\end{problem} 

\begin{problem}{2} Floats
 \subsubsection*{i}
\begin{align*}
0\: 110\: 11111 &= 2^(2^2+2^1 – 3) x 1.11111 \\
&= 1111.112\\
&= 2^4 + 2^3 +\dots{}+2^{-2}\\
&= 15.75_10
\end{align*}

 \subsubsection*{ii}
\begin{align*}
0\: 001\: 00001&= 2^{1-3} * 1.00001\\
&= .0100001_2\\
&= 2^{-2} + 2^{-7 }\\
&= 0.2578125_10
\end{align*}

 \subsubsection*{iii}
\begin{align*}
0\: 000\: 11111&= 2^{1-3} * 0.11111\\
&= 0.0011111_2\\
&= 2^{-3} +\dots{}+ 2^{-7}\\
&= 0.2421875_10
\end{align*}

 \subsubsection*{iv}
\begin{align*}
0\: 000\: 00001&= 2^{1-3} * 0.00001\\
&= 0.0000001_2\\
&= 2^{-7}\\
&= 0.0000078125_10
\end{align*}
\end{problem}


\begin{problem}{3} Floating Point Binary
\begin{center}
\begin{tabular}{ |c|c|c|c|c| } 
 \hline
Number	&Number in Binary&	Exponential&	Exponential Binary&	Binary Float Representation\\\hline
0	& 0		& 0		& 000	& 0\: 000\: 00000\\
1	& 1		& 0+3	& 011	& 0\: 011\: 00000\\
2	& 10		& 1+3	& 100	& 0\: 100\: 00000\\
3	& 11		& 1+3	& 100	& 0\: 100\: 10000\\
4	& 100	& 2+3	& 101	& 0\: 101\: 00000\\
5	& 101	& 2+3	& 101	& 0\: 101\: 01000\\
6	& 110	& 2+3	& 101	& 0\: 101\: 10000\\
7	& 111	& 2+3	& 101	& 0\: 101\: 11000\\
8	& 1000	& 3+3	& 110	& 0\: 110\: 00000\\
9	& 1001	& 3+3	& 110	& 0\: 110\: 00100\\
10	& 1010	& 3+3	& 110	& 0\: 110\: 01000\\
11	& 1011	& 3+3	& 110	& 0\: 110\: 01100\\
12	& 1100	& 3+3	& 110	& 0\: 110\: 10000\\
13	& 1101	& 3+3	& 110	& 0\: 110\: 10100\\
14	& 1110	& 3+3	& 110	& 0\: 110\: 11000\\
15	& 1111	& 3+3	& 110 	& 0\: 110\: 11100\\

 \hline
\end{tabular}
\end{center}
\end{problem}
\end{document}