\documentclass[12pt]{article}
\usepackage[margin=1in]{geometry} 
\usepackage{amsmath,amsthm,amssymb,amsfonts}
\usepackage{listings}
 
\newcommand{\N}{\mathbb{N}}
\newcommand{\Z}{\mathbb{Z}}
 
\newenvironment{problem}[2][Problem]{\begin{trivlist}
\item[\hskip \labelsep {\bfseries #1}\hskip \labelsep {\bfseries #2.}]}{\end{trivlist}}
%If you want to title your bold things something different just make another thing exactly like this but replace "problem" with the name of the thing you want, like theorem or lemma or whatever
 
\begin{document}
 
%\renewcommand{\qedsymbol}{\filledbox}
%Good resources for looking up how to do stuff:
%Binary operators: http://www.access2science.com/latex/Binary.html
%General help: http://en.wikibooks.org/wiki/LaTeX/Mathematics
%Or just google stuff
 
\title{Assignment 2\\ MEEN 357}
\author{Jacob Hartzer}
\maketitle
 
\section*{Task 4}

\begin{lstlisting}
 format long e;
(0.6 + 0.6 + 0.6) - 1.8

ans = 

	-2.220446049250313e-16

\end{lstlisting}


This is not the result that one would expect from the perspective of mathematics. From a math point of view, the answer is clearly exactly zero. However, these floating point numbers aren’t representable as a simple, finite string of binary code. Therefore, there is rounding during each of the above operations. The total rounding is summarized by the above non-zero answer.

\end{document}